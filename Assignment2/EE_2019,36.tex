\documentclass{article}
\title{GATE}
\usepackage{circuitikz}
\usepackage{siunitx}
\usepackage{setspace}
\usepackage{gensymb}
\usepackage{xcolor}
\usepackage{caption}
%\usepackage{subcaption}
\doublespacing
\singlespacing
\usepackage[none]{hyphenat}
\usepackage{amssymb}
\usepackage{relsize}
\usepackage[cmex10]{amsmath}
\usepackage{mathtools}
\usepackage{amsmath}
\usepackage{commath}
\usepackage{amsthm}
\interdisplaylinepenalty=2500
%\savesymbol{iint}
\usepackage{txfonts}
%\restoresymbol{TXF}{iint}
\usepackage{wasysym}
\usepackage{amsthm}
\usepackage{mathrsfs}
\usepackage{txfonts}
\let\vec\mathbf{}
\usepackage{stfloats}
\usepackage{float}
\usepackage{cite}
\usepackage{cases}
\usepackage{subfig}
%\usepackage{xtab}
\usepackage{longtable}
\usepackage{multirow}
%\usepackage{algorithm}
\usepackage{amssymb}
%\usepackage{algpseudocode}
\usepackage{enumitem}
\usepackage{mathtools}
%\usepackage{eenrc}
%\usepackage[framemethod=tikz]{mdframed}
\usepackage{listings}
%\usepackage{listings}
\usepackage[latin1]{inputenc}
%%\usepackage{color}{   
%%\usepackage{lscape}
\usepackage{textcomp}
\usepackage{titling}
\usepackage{hyperref}
%\usepackage{fulbigskip}   
\usepackage{tikz}
\usepackage{graphicx}
\lstset{
  frame=single,
  breaklines=true
}
\let\vec\mathbf{}
\usepackage{enumitem}
\usepackage{graphicx}
\usepackage{siunitx}
\let\vec\mathbf{}
\usepackage{enumitem}
\usepackage{graphicx}
\usepackage{enumitem}
\usepackage{tfrupee}
\usepackage{amsmath}
\usepackage{amssymb}
\usepackage{mwe} % for blindtext and example-image-a in example
\usepackage{wrapfig}
\graphicspath{{figs/}}
\providecommand{\mydet}[1]{\ensuremath{\begin{vmatrix}#1\end{\vmatrix}}}
\providecommand{\myvec}[1]{\ensuremath{\begin{bmatrix}#1\end{\bmatrix}}}
\providecommand{\cbrak}[1]{\ensuremath{\left\{#1\right\}}}
\providecommand{\brak}[1]{\ensuremath{\left(#1\right)}}
\begin{document}
\maketitle
\begin{enumerate}
	\item
In the circuit shown below, $X$ and $Y$ are digital inputs, $Z$ is a digial output,The equivalent circuit is a \\  
		\begin{circuitikz}[american, scale=1][h!]
    % Inputs
    \draw (-2,1.5) node[anchor=east] (x) {X};
    \draw (-2,-1.5) node[anchor=east] (y) {Y};
    
    % Wires before NOT gates
    \draw (x) -- ++(0.5,0) coordinate (xwire);
    \draw (y) -- ++(0.5,0) coordinate (ywire);
    \draw (y) -- ++(0.7,0) coordinate (y1wire);
    % NOT Gates
    \draw (xwire) -- ++(0.5,0) node[not port, anchor=in] (not1) {};
    \draw (ywire) -- ++(0.5,0) node[not port, anchor=in] (not2) {};
    
    % AND Gates
    \draw (not2.out) -- ++(1.5,0.25) node[and port, anchor=in 2] (and2) {};
    \draw (xwire) |- (and2.in 1);
    
    \draw (not1.out) -- ++(1.5,0.25) node[and port, anchor=in 2] (and1) {};
    \draw (y1wire) |- (and1.in 1);
    
    % OR Gate
    \draw (and1.out) -- ++(1.5,0) node[or port, anchor=in 1] (or1) {};
    \draw (and2.out) -- ++(1.5,0) |- (or1.in 2);
    
    % Output
    \draw (or1.out) -- ++(0.5,0) node[anchor=west] (z) {Z};

\end{circuitikz}
\begin{enumerate}
\item
NAND gate
\item
NOR gate
\item
XOR gate
\item
XNOR gate
\end{enumerate}
\end{enumerate}
\end{document}

